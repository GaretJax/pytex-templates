% Use UTF-8 for plain tex files
\usepackage[utf8x]{inputenc}

% Advanced math support
\usepackage{amsmath}
\usepackage{MnSymbol}

% Set the margin from the page side
\usepackage[margin=3cm]{geometry}

% Support for internal links
\usepackage{hyperref}
\hypersetup{
    colorlinks=true,
    linkcolor=black,
    citecolor=black,
    filecolor=black,
    urlcolor=black,
}

% Advanced automatic references
\usepackage{varioref}

% Leave out page numbers on empty pages
\let\origdoublepage\cleardoublepage
\renewcommand{\cleardoublepage}{%
  \clearpage%
  {\pagestyle{empty}\origdoublepage}%
}

% Don't indent paragrpahs, instead separate them
\usepackage{parskip}
\setlength{\parskip}{5mm plus2mm minus3mm}
\setlength{\parindent}{0cm}

% Use alternative font (see http://www.tug.dk/FontCatalogue/ for alternatives)
\usepackage{cmbright}
\renewcommand*\familydefault{\sfdefault} % Set the default font to be sans-serif
\usepackage[T1]{fontenc}

% Set line height multiplicity
\linespread{1.05}

% Allow URL typesetting
\usepackage{url}

% Allow email typesetting
\newcommand{\email}[1]{%
	\href{mailto:#1}{\texttt{#1}}%
}

% Allow advanced list typesetting
\usepackage{enumitem}

% Customize captions appearances 
\usepackage[margin=1cm]{caption}
\DeclareCaptionFormat{listing}{{\em #1#2#3}}
\captionsetup{
	format=listing,
	font={small},
}

% Enable advanced style specifications for the basic listing environments
% (enumerate, itemize, description)
\usepackage{enumitem}

% Customize headers and footers
\usepackage{fancyhdr}

% Includegraphics support
\usepackage{graphicx}

% Allows to change text color
\usepackage[usenames,dvipsnames,table]{xcolor}

% Assign the LastPage label to the last page
\usepackage{lastpage}

% Help better structure documents
\newcommand{\content}{%
	\cleardoublepage%
	\setcounter{page}{1}
}

% Enable the creation of appendices
\usepackage{appendix}

% Support for lists
\usepackage{etoolbox}
\usepackage{fp}

% Advanced templating support
\newcommand{\settitle}[2]{%
	\title{#1 -- #2}
	\def\doctitle{#1}
	\def\subtitle{#2}
}
\newcounter{mseaindex}
\newcounter{mseacount}
\newcommand{\addauthor}[1]{%
	\listadd{\authors}{#1}
	\stepcounter{mseacount}
}
\newcommand{\fmtauthor}[1]{#1}
\newcommand{\lineslist}[1]{#1\\}
\newcommand{\authorlines}{%
	\setcounter{mseaindex}{0}%
	\FPdiv{\limit}{\arabic{mseacount}}{2}%
	\FPround{\limit}{\limit}{0}%
	\renewcommand*{\do}[1]{%
		\ifnumequal{\value{mseaindex}}{0}{}{\ifnumequal{\value{mseaindex}}{\limit}{}{, }}%
		\ifnumequal{\value{mseaindex}}{\limit}{\\[0.7mm]\fmtauthor{##1}}{\fmtauthor{##1}}%
		\stepcounter{mseaindex}%
	}%
	\dolistloop{\authors}%
}

% Set layout lenghts
\setlength{\headheight}{8mm}
\setlength{\footskip}{1.5cm}
\addtolength{\textheight}{-.5cm}

% Set titles whitespace
\usepackage{titlesec}
\titlespacing{\section}{0mm}{8mm}{1mm}
\titlespacing{\subsection}{0mm}{3mm}{-2mm}
\titlespacing{\subsubsection}{0mm}{2mm}{-1mm}

% Enable code listings and set default options
\usepackage{listings}
\usepackage{courier} % Use Adobe Courier instead of Computer Modern Typewriter
                     % for monospaced text

\usepackage{hyphenat}

\newsavebox\leftarrowbox
\sbox\leftarrowbox{\raisebox{-.65mm}{\hspace*{0.3mm}\footnotesize\textcolor{gray}{$\rhookleftarrow$}}}
\makeatletter
\renewcommand*{\BreakableSlash}{%
  \leavevmode
  \prw@zbreak
  \discretionary{\usebox\leftarrowbox}{}{}%
  \prw@zbreak
}
\makeatother

\lstloadlanguages{python,bash,sh,xml}
\lstset{
	language=python,
	basicstyle=\footnotesize\ttfamily,
	stringstyle=\color{OliveGreen},
	numbers=left,
	numberstyle=\color{gray}\tiny,
	commentstyle=\color{magenta},
	keywordstyle=\color{MidnightBlue}\bfseries,
	frame=bt,
	rulecolor=\color{lightgray},
	numbersep=7pt,
	escapechar=¶,
	extendedchars=true,
	captionpos=t,
	breaklines=true,
	showspaces=false,
	showtabs=false,
	tabsize=4, 
	xleftmargin=5pt,
	framexleftmargin=5pt,
	framexrightmargin=0pt,
	framexbottommargin=0pt,
	showstringspaces=false,
}

% TODO: Create macros for lstlisting with or without line numbers

\lstdefinelanguage{diff}{
	morecomment=[f][\color{blue}]{@@},     % group identifier
	morecomment=[f][\color{Red}]-,         % deleted lines
	morecomment=[f][\color{OliveGreen}]+,  % added lines
	morecomment=[f][\color{Gray}]{---},    % Diff header lines (must appear after +,-)
	morecomment=[f][\color{Gray}]{+++},
}
  % Diff files highlighting
\lstdefinelanguage{prompt}{
	morecomment=[l][\color{Black}]{\#},
	morecomment=[l][\color{Black}]{\$},
	morecomment=[f]{\#},
}

\lstnewenvironment{rootprompt}[1][caption=,label=]{%
	\lstset{
		language=prompt,
		numbers=none,
		basicstyle=\color{BrickRed}\footnotesize\ttfamily,
		xleftmargin=6pt,
		framexleftmargin=6pt,
		framextopmargin=.5pt,
		framexbottommargin=1pt,
		rulecolor=\color{verylightgray},
		backgroundcolor=\color{verylightgray},
		#1
	}
}{\vspace*{-2mm}}

\lstnewenvironment{userprompt}[1][caption=,label=]{%
	\lstset{
		language=prompt,
		numbers=none,
		basicstyle=\color{RoyalBlue}\footnotesize\ttfamily,
		xleftmargin=0pt,
		xrightmargin=0pt,
		framexleftmargin=6pt,
		framexrightmargin=6pt,
		framextopmargin=0pt,
		framexbottommargin=0pt,
		rulecolor=\color{verylightgray},
		backgroundcolor=\color{verylightgray},
		#1
	}
}{\vspace*{-2mm}}

\lstnewenvironment{cmdresult}[1][caption=,label=,language=]{%
	\lstset{
		numbers=none,
		xleftmargin=0pt,
		xrightmargin=0pt,
		framexleftmargin=6pt,
		framexrightmargin=6pt,
		framextopmargin=0pt,
		framexbottommargin=0pt,
		language=,
		frame=b,
		#1
	}
	\vspace*{-1.2mm}
}{\vspace*{-2mm}}


\lstnewenvironment{edit}[1][caption=,label=,language=bash]{%
	\lstset{numbers=left,xleftmargin=17pt,framexleftmargin=17pt,numbersep=10pt,#1}
}{\vspace*{-2mm}}
  % Special formatting for command prompts
\lstdefinelanguage{javascript}{
	keywords={typeof, new, true, false, catch, function, return, null, catch, switch, var, if, in, while, do, for, else, case, break, try},
	ndkeywords={class, export, boolean, throw, implements, import, this},
	sensitive=false,
	comment=[l]{//},
	morecomment=[s]{/*}{*/},
	morestring=[b]',
	morestring=[b]"
}
  % Javascript highlighting 

% Enable the use of footnotes in section headings
% http://www.tex.ac.uk/cgi-bin/texfaq2html?label=ftnsect
\usepackage[stable]{footmisc}

% Limit the TOC dept to sections and subsections
\setcounter{tocdepth}{2}

\usepackage{multirow} % Enable columns spawning multiple rows
\usepackage{tabularx}
\usepackage{booktabs} % Provides {top|middle|bottom}rule
\usepackage{longtable} % Support for tables spawning multiple pages,
                       % captions and footnotes.
                       % Does not work in multicolumn environments
% Define new tabularx column types:
%  - R: streteched right aligned
%  - C: stretched centered
\newcolumntype{R}{>{\raggedleft\arraybackslash}X}%
\newcolumntype{C}{>{\centering\arraybackslash}X}%
\renewcommand{\arraystretch}{1.3} % Set row height multiplicator
\usepackage{threeparttable} % Manual footnotes counter

% Provide support for (indented) unnumbered chapter and section titles
\usepackage{tocloft}
\newcommand{\uchapter}[1]{
	\chapter*{#1}
	\addcontentsline{toc}{chapter}{\hspace{\cftchapnumwidth}#1}
}
\newcommand{\usection}[1]{
	\plainsection*{#1}
	\addcontentsline{toc}{section}{\hspace{\cftsecnumwidth}#1}
}

% Use subfloats for figures and tables (must be loaded after the caption and tocloft packages)
\usepackage[lofdepth,lotdepth]{subfig}

% Color definitions
\definecolor{highlightyellow}{RGB}{255,255,140}
\definecolor{mselogogray}{RGB}{96,101,109} % Color definition for the MSE logo
\definecolor{verylightgray}{RGB}{240,240,240}

% Enable highlighting using \HighlightFrom and \HighlightTo
\usepackage{classes/highlighter}
\tikzset{highlighter/.style={highlightyellow, line width=0.75\baselineskip}}

% Advanced inclusion support for external documents
\usepackage{pdfpages}

% Support for conditional structures
\usepackage{ifthen}

\let\plainsection\section
\renewcommand\section[2][\_nolabel]{
	\plainsection{#2}
	\ifthenelse{\equal{#1}{\_nolabel}}{}{\label{sec:#1}}
}

% Support for todo entries
\newcommand{\todo}[1]{
	{\color{lightgray} \rule{\linewidth}{.5pt}}
	\textbf{TODO} \hspace{3pt} #1\\[-6pt]
	{\color{lightgray} \rule{\linewidth}{.5pt}}
}
